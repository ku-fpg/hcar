% KansasLava-AK.tex
\begin{hcarentry}{Kansas Lava}
\label{klava}
\report{Andy Gill}%05/14
\participants{Bowe Neuenschwander}
\status{ongoing}
\makeheader

Kansas Lava is a Domain Specific Language (DSL) for expressing
hardware descriptions of computations, and is hosted inside the
language Haskell. Kansas Lava programs are descriptions of specific hardware
entities, the connections between them, and other computational abstractions
that can compile down to these entities. Large circuits have been successfully
expressed using Kansas Lava, and Haskell's powerful abstraction mechanisms, as
well as generic generative techniques, can be applied to good effect to provide
descriptions of highly efficient circuits.

\begin{itemize}

\item The Fabric monad is now a Monad transformer.
The Fabric monad historically provided access to named input/output ports,
and now also provides named variables, implemented by ports that loop back on
themselves. This additional primitive capability allows for a {\em typed\/}
state machine monad.
This design gives an elegant stratospheric pattern: purely functional circuits using streams;
a monad for layout over {\em space\/}; and a monad for state generation,
that acts over {\em time\/}.

\item 
On top of the Fabric monad, we are implementing an atomic transaction
layer, which provides a BSV-like interface, but in Haskell. An initial
implementation has been completed, and this is being reworked to include
BSV's Ephemeral History Registers.
\end{itemize}

\FurtherReading
  \url{http://www.ittc.ku.edu/csdl/fpg/Tools/KansasLava}
\end{hcarentry}
