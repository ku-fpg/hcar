% KansasLava-AK.tex
\begin{hcarentry}{Kansas Lava}
\label{klava}
\report{Andy Gill}%11/10
\participants{Tristan Bull, Andrew Farmer, Ed~Komp}
\status{ongoing}
\makeheader

Kansas Lava is a modern implementation of a hardware description language
that uses functions to express hardware components,
and leverages the abstractions in Haskell to build complex circuits. 
Lava, 
the given name for a family of Haskell based hardware description libraries~\cref{ylava}\cref{chalmers},
is an idiomatic way of expressing hardware in Haskell which allows for simulation and
synthesis to hardware.

Though there has been no public release (yet), we have made considerable
progress with Kansas Lava. We have generated several large telemetry circuits,
which have been synthesized and tested on real hardware, running at speeds
comparable to other implementation techniques.
A talk about internals of Kansas Lava was presented by Andrew Farmer at the
Haskell implementors workshop in October, and the talk and slides are available online.

Jun Inoue from Rice University visited CSDL for October and November, to help
connect his ``staging'' work with the Kansas Lava work.
%
A release of Kansas Lava release was planned for the end of 2010.

\FurtherReading
  \url{http://www.ittc.ku.edu/csdl/fpg/Tools/KansasLava}
\end{hcarentry}
