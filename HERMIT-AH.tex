% HERMIT-AH.tex
\begin{hcarentry}{HERMIT}
\label{HERMIT}
\report{Andy Gill}%11/11
\participants{Andy Gill, Andrew Farmer, Ed~Komp, Neil Sculthorpe}
\status{active}
\makeheader

The Haskell Equational Reasoning Model-to-Implementation Tunnel
(HERMIT) is an NSF-funded project being run at KU \cref{ukansas} to improving the
Applicability of Haskell-Hosted Semi-Formal Models to High Assurance
Development. Specifically, HERMIT will use the worker/wrapper
transformation, a Haskell-hosted DSL, and a new refinement UI to
perform rewrites directly on Haskell Core, the GHC internal
representation.

This project is a substantial case study into the application of
worker/wrapper on larger examples. In particular, we want to
demonstrate the equivalences between efficient Haskell programs, and
their clear, specification-style Haskell counterparts. In doing so,
there are several open problems, including refinement scripting and
management scaling issues, data representation and presentation
challenges, and understanding the theoretical boundaries of the
worker/wrapper transformation.

The project started in Spring 2012, and is expected to run
for 2 years. Neil Sculthorpe, who got his PhD from
the University of Nottingham, has joined as the senior member of the project,
and Andrew Framer and Ed Komp round
out the team. We have already reworked the Domain Specific Language
KURE as our basis of our rewrite capabilities, and constructed the
rewrite kernel. The entire system uses the GHC plugin architecture,
and we have small examples successfully being transformed through
a simple REPL. A web-based API is being constructed, and an
Android version is planned.
We hope to write up a detailed introduction of our architecture
and implementation for the Haskell Symposium.

\FurtherReading
  \url{http://www.ittc.ku.edu/csdl/fpg/Tools/HERMIT}
\end{hcarentry}
