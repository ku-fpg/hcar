% HERMIT-AH.tex
\begin{hcarentry}[updated]{HERMIT}
\label{HERMIT}
\report{Andy Gill}%05/12
\participants{Andy Gill, Andrew Farmer, Ed~Komp, Neil Sculthorpe,
Adam Howell, Robert Blair, Ryan Scott, Patrick Flor, Michael Tabone}
\status{active}
\makeheader

The Haskell Equational Reasoning Model-to-Implementation Tunnel
(HERMIT) is an NSF-funded project being run at KU \cref{ukansas}, which aims to improve the
applicability of Haskell-hosted Semi-Formal Models to High Assurance Development.
Specifically, HERMIT will use: a Haskell-hosted DSL; the Worker/Wrapper Transformation;
and a new refinement user interface to perform rewrites directly on Haskell Core, the GHC internal representation.

This project is a substantial case study of the application of
Worker/Wrapper on larger examples. In particular, we want to
demonstrate the equivalences between efficient Haskell programs, and
their clear specification-style Haskell counterparts. In doing so
there are several open problems, including refinement scripting and
managing scaling issues, data representation and presentation
challenges, and understanding the theoretical boundaries of the
worker/wrapper transformation.

The project started in Spring 2012, and is expected to run for two years.
The HERMIT team currently consists of
one assistant professor, %Andy
one research engineer, %Ed
one post-doc, %Neil
one PhD student, %Drew
one Masters student, %Jan
two undergraduates,
and three student and ex-student volunteers. % Nick + 2 undergrads?
%
% Neil Sculthorpe, who got his PhD from the University of Nottingham in 2011, has joined as a senior member of the project,
% and Andrew Farmer and Ed Komp round out the team.
%

We have reworked KURE (\url{http://www.haskell.org/communities/11-2008/html/report.html#sect5.5.7}), a Haskell-hosted DSL for strategic programming, as the basis of our rewrite capabilities, and constructed the rewrite kernel making use of the GHC Plugins architecture.
As for interfaces to the kernel, we currently have a command-line REPL, we are constructing a web-based API, and an Android version is planned.
A detailed introduction to the HERMIT architecture and implementation can be found in our Haskell Symposium 2012 paper.
We have used HERMIT to successfully mechanize about a dozen small examples of program transformations, drawn from the literature on techniques such as concatenate vanishes, tupling transformation, and worker/wrapper.
A discussion of our experiences mechanizing these examples can be found in our IFL 2012 paper.
HERMIT has also been used to optimize generic traversals written using Scrap Your Boilerplate, see \cref{sec:optimisingGenerics} for details.

Funded by the NSF REU initiative and ITTC (our research center),
we are also working on an Android application interface for HERMIT,
where gestures can be used to manipulate Haskell Core programs.
Five KU undergraduates are working on the Android application.

\FurtherReading
  \url{http://www.ittc.ku.edu/csdl/fpg/Tools/HERMIT}
\end{hcarentry}
