% HERMIT-AH.tex
\begin{hcarentry}[updated]{HERMIT}
\label{HERMIT}
\report{Andrew Gill}%05/15
\participants{Andrew Farmer, Neil Sculthorpe, Ryan Scott}
\status{active}
\makeheader

The Haskell Equational Reasoning Model-to-Implementation Tunnel
(HERMIT) is an NSF-funded project being run at KU \cref{ukansas}, which aims to improve the
applicability of Haskell-hosted Semi-Formal Models to High Assurance Development.
Specifically, HERMIT uses a Haskell-hosted DSL
and a new refinement user interface to perform rewrites directly on Haskell Core, the GHC internal representation.

In the project we want to
demonstrate the equivalences between efficient Haskell programs, and
their clear specification-style Haskell counterparts. In doing so
there are several open problems, including refinement scripting and
managing scaling issues, data representation and presentation
challenges, and understanding the theoretical boundaries of the
worker/wrapper transformation.

We have reworked KURE, a Haskell-hosted DSL for strategic programming, as the basis of our rewrite capabilities,
and constructed the rewrite kernel making use of the GHC Plugins architecture.
A journal writeup of the KURE internals is available in JFP.
As for interfaces to the kernel,
we currently have a command-line REPL, which we are replacing this summer with a GHCi DSL,
called Black Shell.
We have used HERMIT to successfully mechanize many smaller examples of program transformations,
drawn from the literature on techniques such as concatenate vanishes, tupling transformation, and worker/wrapper.

\FurtherReading
  \url{https://github.com/ku-fpg/hermit}
\end{hcarentry}
