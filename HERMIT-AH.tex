% HERMIT-AH.tex
\begin{hcarentry}{HERMIT}
\label{HERMIT}
\report{Andy Gill}%05/14
\participants{Andrew Farmer, Ed~Komp, Neil Sculthorpe,
Adam Howell, Ryan Scott}
\status{active}
\makeheader

The Haskell Equational Reasoning Model-to-Implementation Tunnel
(HERMIT) is an NSF-funded project being run at KU \cref{ukansas}, which aims to improve the
applicability of Haskell-hosted Semi-Formal Models to High Assurance Development.
Specifically, HERMIT uses a Haskell-hosted DSL
and a new refinement user interface to perform rewrites directly on Haskell Core, the GHC internal representation.

This project is a substantial case study of the application of
Worker/Wrapper. In particular, we want to
demonstrate the equivalences between efficient Haskell programs, and
their clear specification-style Haskell counterparts. In doing so
there are several open problems, including refinement scripting and
managing scaling issues, data representation and presentation
challenges, and understanding the theoretical boundaries of the
worker/wrapper transformation.

We have reworked KURE (\url{http://www.haskell.org/communities/11-2008/html/report.html#sect5.5.7}),
a Haskell-hosted DSL for strategic programming, as the basis of our rewrite capabilities,
and constructed the rewrite kernel making use of the GHC Plugins architecture.
A journal writeup of the KURE internals has been submitted to JFP, and is available
on the group webpage.
As for interfaces to the kernel,
we currently have a command-line REPL, and an Android version is under development.
We have used HERMIT to successfully mechanize many smaller examples of program transformations,
drawn from the literature on techniques such as concatenate vanishes, tupling transformation, and worker/wrapper.

We are scaling up our capabilities, and working on larger examples,
including the mechanization of improvements to the classical game of life.
We also submitted a paper to the Haskell Symposium 2014 with a description of
our support for equational reasoning in HERMIT, including a mechanization
of Bird's ``Making a Century''.

HERMIT is also being used in three projects.
\begin{itemize}
\item Michael Adams used HERMIT to mechanize the optimization of scrap
  your boilerplate generics, leading to execution speeds that were as
  fast as hand-optimized code \cref{sec:optimisingGenerics}. The
  result was published in PEPM, where it was awarded a best paper
  award.
\item Andrew Farmer has used HERMIT to implement a custom GHC
  optimization pass which enables fusion of nested streams within the
  Stream Fusion framework.  This work was also published in PEPM.
\item Finally, Conal Elliott is working on typed reification of Core using HERMIT.
\end{itemize}

\FurtherReading
  \url{https://github.com/ku-fpg/hermit}
\end{hcarentry}
