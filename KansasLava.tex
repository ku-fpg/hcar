% KansasLava-AK.tex
\begin{hcarentry}{Kansas Lava}
\label{klava}
\report{Andy Gill}%11/11
\participants{Andy Gill, Andrew Farmer, Ed~Komp, Bowe Neuenschwander,
Garrin Kimmell (University of Iowa)}
\status{ongoing}
\makeheader

Kansas Lava is a Domain Specific Language (DSL) for expressing
hardware descriptions of computations, and is hosted inside the
language Haskell. Kansas Lava programs are descriptions of specific hardware
entities, the connections between them, and other computational abstractions
that can compile down to these entities. Large circuits have been successfully
expressed using Kansas Lava, and Haskell's powerful abstraction mechanisms, as
well as generic generative techniques, can be applied to good effect to provide
descriptions of highly efficient circuits.
Kansas Lava draws considerably from Xilinx Lava~%\cref{ylava} 
(\url{http://www.haskell.org/communities/11-2010/html/report.html#sect3.7})
and
Chalmers Lava~\cref{chalmers}.

The release of Kansas Lava, version 0.2.4, happened in early November.
Based round this release, there are a number of resources for users,
including a (draft) tutorial, and a youtube channel with
walkthroughs of our Lava in use.

On top of Kansas Lava, we are developing Kansas Lava Cores, which was released
on hackage at the same time as Kansas Lava. In hardware, a core is a component
that can be realized as a circuit, typically on an FPGA. Kansas Lava Cores
contains about a dozen cores, and basic board support for Spartan3e,
as well as an emulator for the Spartan3e.

Using various components provided as Kansas Lava Cores, we are developing the
$\lambda$-bridge~\cref{lambdabridge}, with implementations in Haskell and
Kansas Lava of a simple protocol stack for communicating with FPGAs. We have
early prototypes working, and implementation in Kansas Lava continues.

Finally, we are working on a Lava idiom called a {\tt Patch}, which is a Kansas
Lava component interface that uses types to declare protocols and handshakes
needed and used. Most of components in the Kansas Lava Cores are instances of
our {\tt Patch} idiom. There is a PADL 2012 paper describing {\tt Patch},
including the design and implementation of a controller for an ST7066U-powered
LCD display.

Tristan Bull graduated in May 2011 with an MS. His MS thesis was about using
Kansas Lava. Congratulations Tristan!

\FurtherReading
\begin{compactitem}
\item
  \url{http://www.ittc.ku.edu/csdl/fpg/Tools/KansasLava}\\
\item
  \url{http://www.youtube.com/playlist?list=PL211F8711E3B3DF9C}  
\end{compactitem}
\end{hcarentry}
