%\documentclass{scrreprt}
\documentclass{article}
\usepackage{paralist}
\usepackage{graphicx}
\usepackage{hcar}

\begin{document}

\begin{hcarentry}{Kansas Lava}
\report{Andy Gill}
\status{Ongoing}
\participants{Andy Gill, Tristan Bull, Andrew Farmer, Ed~Komp}% optional
\makeheader

Kansas Lava is a modern implementation of a hardware description language
that uses functions to express hardware components,
and leverages the abstractions in Haskell to build complex circuits. 
Lava, 
the given name for a family of Haskell based hardware description libraries,
is an idiomatic way of expressing hardware in Haskell which allows for simulation and
synthesis to hardware.

Though there has been no public release (yet), we have made considerable
progress with Kansas Lava. We generate several large telemetry circuits,
which have been synthesized and tested on real hardware, running at speeds
comparable to other implementation techniques.
The talk about internals of Kansas Lava was presented by Andrew Farmer at the
Haskell implementors workshop in October, and the talk and slides are available online.

Jun Inoue from Rice University visited for CSDL for October and November, to help
connect his ''staging'' work with the Kansas Lava work.

A release of Kansas Lava release is planned for the end of the year.

\FurtherReading
  \url{http://www.ittc.ku.edu/csdl/fpg/Tools/KansasLava}

\end{hcarentry}

\begin{hcarentry}{ChalkBoard}
\report{Andy Gill}
\status{Ongoing}
\participants{Kevin Matlage, Andy Gill}% optional
\makeheader

ChalkBoard is a domain specific language for describing images. 
The language is uncompromisingly functional
and encourages the use of modern functional idioms.
The novel contribution of ChalkBoard is that it uses off-the-shelf
graphics cards to speed up rendering of our functional description.
We always intended to use ChalkBoard to animate educational
videos, as well as for processing streaming videos.
Since the last HCAR report, we've added a new animation language, based round a new applicative
functor, \verb|Active|. It has been called Functional Reactive Programming,
without the reactive part! The paper``Every Animation Should Have a Beginning, a Middle, and an End'' talks
about this addition. 

A release is scheduled for November 2010.

\FurtherReading
  \url{http://www.ittc.ku.edu/csdl/fpg/Tools/ChalkBoard}

\end{hcarentry}

\begin{hcarentry}{Functional Programming at KU}
\report{Andy Gill}
\status{Ongoing}
\makeheader

\includegraphics[width=100pt]{jh2.jpg}

Functional Programming remains active at KU and 
the Computer Systems Design Laboratory in ITTC.
The System Level Design Group (lead by Perry Alexander)
and the Functional Programming Group (lead by Andy Gill)
together form the core functional programming initiative at KU.
Apart from Kansas Lava and ChalkBoard, there are many other
FP and Haskell related things going on.

\begin{itemize}
\item We are developing a Haskell version of HOL.
Traditionally, members of the higher-order logic theorem (HOL) proving family have
been implemented in the Standard ML programming language or one of its derivatives.
HaskHOL aims to break with tradition by implementing a lightweight HOL theorem prover
library as a Haskell hosted domain specific language. Based on the HOL Light logical
system, HaskHOL aims to provide the ability for Haskell users to reason about their
code directly without having to transform it or otherwise export it to an external
tool. For details talk to Evan Austin.

\item We are actively working on enabling {\em Type-Directed Specification Refinement in Rosetta\/}. Rosetta is a specification language that focuses on the interaction between different domains, such as state-based and signal-based domains.  With dependent types, first-class types, and reflection, there are many areas where a traditional all-or-nothing typing analysis would be impractical--especially when considering that specifications are likely written at first in a high-level, incomplete fashion.  This project uses InterpreterLib  and various Rosetta analysis tools to define a typing analysis that attempts to extract typing information, constraints, and errors to present to the user, in order to guide the specification refinement process.  It is in the early stages of development, but may eventually link up with HaskHOL to discharge some TCC's. For details talk to Mark Snyder.

\item We are developing a library in Haskell for processing Rosetta specifications.
A current focus is the modularity and re-use of distinct processing
elements, such as type-checking, partial evaluation, and reasoning
assistants. Mutually defined elements that are more convenient to
consider as distinct interact via a reactive monadic computation, so
the two elements' code can be managed as separate packages. Also, our
principal specification representation use functors and type-level
fixed points to achieve extensibility and generic programming. The
goal of the library is to provide to a tight and graduated interface
to the basic processing elements, so that the users may incorporate
the most appropriate basic elements when implementing their own, more
domain-specific Rosetta processors. For details talk to Nick Frisby.

\item We are working with other functional programming groups round the world (University of Iowa, St. Andrews, Heriot Watt, Halmstad Universit, and of course Chalmers) to share our common experiences with using FPGA board, and generating VHDL.
So far, we have chosen and purchased specific Xilinx boards, and have a design for a so called ``$\lambda$-bridge'' between our UNIX invocation infrastructures and our FPGA boards. The idea is we can share experiences, for the sake of 
being able to spend more time working on FP issues, and bringing FP ideas to hardware related problems.
\end{itemize}

\FurtherReading
  The Functional Programming Group
    \url{http://www.ittc.ku.edu/csdl/fpg}.

  CSDL website: \url{https://wiki.ittc.ku.edu/csdl/Main_Page}
\end{hcarentry}


\end{document}
