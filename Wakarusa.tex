\begin{hcarentry}{Wakarusa}
\label{Wakarusa}
\report{Andy Gill}%05/15
\participants{Andy Gill, Mark Grebe, Ryan Scott, James Stanton, David Young}
\status{active}
\makeheader

The Wakarusa project is a domain-specific language toolkit,
that makes domain-specific languages easier to deploy in
high-performance scenarios. The technology is going to be initially
applied to two types of high-performance platforms, GPGPUs and
FPGAs. However, the toolkit will be general purpose, and we expect the
result will also make it easier to deploy DSLs in situations where
resource usage needs to be well-understand, such as cloud resources
and embedded systems. The project is funded by the NSF.

Wakarusa is a river just south of Lawrence, KS, where the main campus
of the University of Kansas is located. Wakarusa is approximately
translated as ``deep river'', and we use deep embeddings a key
technology in our DSL toolkit. Hence the project name Wakarusa.

A key technical challenge with syntactic alternatives to deep embeddings
is knowing when to stop unfolding. We are using a new design pattern,
called the remote monad, which allows a monad to be virtualized, and
run remotely, to bound our unfolding. 
%
We have already used remote monads for graphics (Blank Canvas),
hardware bus protocols ($\lambda$-bridge), and a driver for MineCraft.
Using the remote monad design pattern, and HERMIT, we are developing
a translation framework that translates monadic Haskell to GPGPUs (building on
accelerate), and monadic Haskell to Hardware (building on Kansas Lava),
and monadic imperative Haskell to Arduino C.

\FurtherReading
\begin{compactitem}
\item
  \url{https://github.com/ku-fpg/wakarusa}
\item
  \url{http://ku-fpg.github.io/research/wakarusa/}
\end{compactitem}
\end{hcarentry}
