% KansasLava-AK.tex
\begin{hcarentry}{$\lambda$-Bridge}
\label{lambdabridge}
\report{Andy Gill}%11/10
\participants{Andy Gill, Bowe Neuenschwander, Patrick Miller, Ed~Komp}
\status{ongoing}
\makeheader

The $\lambda$-bridge effort provides enabling technology for using functional
programming on FPGA fabrics and boards. The majority of the artifacts are
shared documentation of ways to use FPGA board, and libraries (software and
hardware) that facilitate the use of FPGAs.
Techniques for {\em programing\/} FPGA boards are well documented, and there
are many online examples and other resources to draw from. Getting data to and
from a new hardware configuration, however, is a problem every bit (pun
intended) as challenging as programming a FPGA in the first place. From
empirical evidence, engineers that need communications with a host processor
write custom VHDL or Verilog for their specific board to solve this problem.
$\lambda$-bridge helps solve this problem.

We are using Kansas Lava to generate various ``Cores'' that provide
a network protocol stack centered
round the simple $\lambda$-bridge protocol, while being generic
about the physical layer.
For example, we support the RS-232 cable and UDP over ethernet, 
and have plans for USB and (where applicable) directly via a motherboard bus. 
The protocol is also implemented in Haskell, to provided the host-side
support.
Using the $\lambda$-bridge is not the fastest way of communicating
with a board, but we hope will be an easy way of getting up and running
with a design.
Between Kansas Lava, Kansas Lava Cores, and $\lambda$-bridge, we plan
to introduce a new generation of functional programmers to the joys
of FPGA programming.

We are especially interested in using $\lambda$-bridge to build
a bridge between the statistics language R, and FPGA board,
in an attempt to speed up some common statical operations
by offshoring the computation to FPGAs.

\FurtherReading
  \url{http://www.ittc.ku.edu/csdl/fpg/Tools/LambdaBridge}
\end{hcarentry}
