% TheRemoteMonadDesignPatte-AT.tex
\begin{hcarentry}[new]{The Remote Monad Design Pattern}
\label{RemoteMonad}
\report{Andrew Gill}%11/15
\participants{Justin Dawson, Mark Grebe, James Stanton, David Young}
\status{active}
\makeheader

The {\bf remote monad design pattern} is a way of making Remote Procedure
Calls (RPCs), and other calls that leave the Haskell eco-system, considerably
less expensive.  The idea is that, rather than directly call a remote
procedure, we instead give the remote procedure call a service-specific
monadic type, and invoke the remote procedure call using a monadic ``send''
function.  Specifically, a {\bf remote monad} is a monad that has its
evaluation function in a remote location, outside the local runtime system.

By factoring the RPC into sending invocation and service name, we can
group together procedure calls, and amortize the cost of the remote
call. To give an example, Blank Canvas, our library for remotely
accessing the JavaScript HTML5 Canvas, has a \verb|send| function,
\verb`lineWidth` and \verb`strokeStyle` services, and our remote monad is called
\verb`Canvas`:

{\footnotesize
\begin{verbatim}
send        :: Device -> Canvas a -> IO a
lineWidth   :: Double             -> Canvas ()
strokeStyle :: Text               -> Canvas ()
\end{verbatim}
}

If we wanted to change the (remote) line width,
the \verb`lineWidth` RPC can be invoked by combining \verb`send`
and \verb`lineWidth`:

{\footnotesize
\begin{verbatim}
send device (lineWidth 10)
\end{verbatim}
}

Likewise, if we wanted to change the (remote) stroke color,
the \verb`strokeStyle` RPC can be invoked by combining \verb`send`
and \verb`strokeStyle`:

{\footnotesize
\begin{verbatim}
send device (strokeStyle "red")
\end{verbatim}
}

The key idea is that remote monadic commands can
be locally combined before sending them to a remote server.
For example:

{\footnotesize
\begin{verbatim}
send device (lineWidth 10 >> strokeStyle "red")
\end{verbatim}
}

The complication is that, in general, monadic commands can return a
result, which may be used by subsequent commands.  For example, if we
add a monadic command that returns a Boolean,

{\footnotesize
\begin{verbatim}
isPointInPath :: (Double,Double) -> Canvas Bool
\end{verbatim}
}

we could use the result as follows:

{\footnotesize
\begin{verbatim}
   send device $ do
      inside <- isPointInPath (0,0)
      lineWidth (if inside then 10 else 2)
      ...
\end{verbatim}
}

The invocation of \verb`send` can also return a value:

{\footnotesize
\begin{verbatim}
  do res <- send device (isPointInPath (0,0))
     ...
\end{verbatim}
}

Thus, while the monadic commands inside \verb`send` are executed in a
remote location, the results of those executions need to be made
available for use locally. 
% This is the remote monad design pattern.

We had a paper in the 2015 Haskell Symposium that discusses these ideas in more detail,
and more recently, we have improved the packet mechanism to include an analog of
the applicate monad structure, allowing for even better bundling. We have also improved
the error handling capabilities. 
These ideas are implemented up in the hackage package remote-monad, which captures
the pattern, and automatically bundled the monadic requests.

\FurtherReading
  \url{http://ku-fpg.github.io/practice/remotemonad}
  \url{https://hackage.haskell.org/package/remote-monad}
\end{hcarentry}
