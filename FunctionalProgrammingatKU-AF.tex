% FunctionalProgrammingatKU-AF.tex
\begin{hcarentry}[section,updated]{Functional Programming at KU}
\label{ukansas}
\report{Andy Gill}%11/12
\status{ongoing}
\makeheader

%**<img width=200 src="./jh2.jpg">
%*ignore
\begin{center}
\includegraphics[width=0.235\textwidth]{html/jh2.jpg}
\end{center}
%*endignore

Functional Programming is vibrant at KU and
the Computer Systems Design Laboratory in ITTC!
The System Level Design Group (lead by Perry Alexander)
and the Functional Programming Group (lead by Andy Gill)
together form the core functional programming initiative at KU.
Apart from Kansas Lava~\cref{klava} and HERMIT~\cref{HERMIT},
there are several other
FP and Haskell related things going on,
primarily in the area of web technologies.

We are interested in providing better support for
interactive applications in Haskell by building on top of existing web technologies,
like the fast Chrome browser, HTML5, and JavaScript. This is motivated
partly by having easy tools to interactively teach programming in Haskell,
and partly by the needs of the HERMIT~\cref{HERMIT} project.

Towards this, we have developed a lightweight web framework called {\tt Scotty}.
Modeled after Ruby's popular Sinatra framework, Scotty is intended to
be a cheap and cheerful way to write RESTful, declarative web applications.
Scotty borrows heavily from the Yesod \cref{yesod} ecosystem, conforming
to the WAI \cref{wai} interface and using the fast Warp \cref{warp} web server
by default. More information can be found at the link below.

On top of {\tt Scotty}, we are building {\tt sunroof},
a deeply-embedded Javascript compiler, allowing for
the handling of arbitrary asynchronous Javascript events
directly on the browser. The initial design and
implementation was written up in  XLDI'12.

Finally, in August 2012 Nicolas Frisby successfully defended his
PhD, and plans to start an internship at MSR Cambridge early
next year. 

\FurtherReading
\begin{compactitem}
\item   The Functional Programming Group:
    \url{http://www.ittc.ku.edu/csdl/fpg}
\item
  CSDL website: \url{https://wiki.ittc.ku.edu/csdl/Main_Page}
\item \url{http://www.ittc.ku.edu/csdl/fpg/Tools/Scotty}
\item \url{http://www.ittc.ku.edu/csdl/fpg/Tools/BlankCanvas}
\end{compactitem}
\end{hcarentry}
