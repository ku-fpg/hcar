% FunctionalProgrammingatKU-AF.tex
\begin{hcarentry}[section]{Functional Programming at KU}
\report{Andy Gill}%11/10
\status{ongoing}
\makeheader

%**<img width=200 src="./jh2.jpg">
%*ignore
\begin{center}
\includegraphics[width=0.235\textwidth]{html/jh2.jpg}
\end{center}
%*endignore

Functional Programming remains active at KU and 
the Computer Systems Design Laboratory in ITTC.
The System Level Design Group (lead by Perry Alexander)
and the Functional Programming Group (lead by Andy Gill)
together form the core functional programming initiative at KU.
Apart from Kansas Lava~\cref{klava}, ChalkBoard~\cref{chalkboard}, 
Lambda Bridge~\cref{lambdabridge} and HERMIT~\cref{HERMIT},
there are several other
FP and Haskell related things going on.

\begin{itemize}

\item We are continuing our development of a lightweight Haskell version of HOL.
Traditionally, members of the higher-order logic theorem (HOL) proving family
have been implemented in the Standard ML programming language or one of its
derivatives. HaskHOL aims to break with tradition by implementing a lightweight
HOL theorem prover library as a Haskell hosted domain specific language. Based
on the HOL Light logical system, HaskHOL aims to provide the ability for Haskell
users to reason about their code directly without having to transform it or
otherwise export it to an external tool. For details talk to Evan Austin.

\item We are developing a library in Haskell for processing Rosetta
specifications. A current focus is the modularity and re-use of distinct
processing elements, such as type-checking, partial evaluation, and reasoning
assistants. Mutually defined elements that are more convenient to consider as
distinct interact via a reactive monadic computation, so the two elements' code
can be managed as separate packages. Also, our principal specification
representation use functors and type-level fixed points to achieve extensibility
and generic programming. The goal of the library is to provide to a tight and
graduated interface to the basic processing elements, so that the users may
incorporate the most appropriate basic elements when implementing their own,
more domain-specific Rosetta processors. For details talk to Nick Frisby.
\end{itemize}

\FurtherReading
\begin{compactitem}
\item   The Functional Programming Group:
    \url{http://www.ittc.ku.edu/csdl/fpg}

\item
  CSDL website: \url{https://wiki.ittc.ku.edu/csdl/Main_Page}
\end{compactitem}
\end{hcarentry}
