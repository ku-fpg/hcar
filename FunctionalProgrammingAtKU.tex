% FunctionalProgrammingatKU-AF.tex
\begin{hcarentry}[section]{Functional Programming at KU}
\report{Andy Gill}%11/10
\status{ongoing}
\makeheader

%**<img width=200 src="./jh2.jpg">
%*ignore
\begin{center}
\includegraphics[width=0.235\textwidth]{html/jh2.jpg}
\end{center}
%*endignore

Functional Programming remains active at KU and 
the Computer Systems Design Laboratory in ITTC.
The System Level Design Group (lead by Perry Alexander)
and the Functional Programming Group (lead by Andy Gill)
together form the core functional programming initiative at KU.
Apart from Kansas Lava~\cref{klava} and ChalkBoard~\cref{chalkboard}, there are many other
FP and Haskell related things going on.

\begin{itemize}
\item We are developing a Haskell version of HOL.
Traditionally, members of the higher-order logic theorem (HOL) proving family have
been implemented in the Standard ML programming language or one of its derivatives.
HaskHOL aims to break with tradition by implementing a lightweight HOL theorem prover
library as a Haskell hosted domain specific language. Based on the HOL Light logical
system, HaskHOL aims to provide the ability for Haskell users to reason about their
code directly without having to transform it or otherwise export it to an external
tool. For details talk to Evan Austin.

\item We are actively working on enabling {\em Type-Directed Specification Refinement in Rosetta\/}. Rosetta is a specification language that focuses on the interaction between different domains, such as state-based and signal-based domains.  With dependent types, first-class types, and reflection, there are many areas where a traditional all-or-nothing typing analysis would be impractical --- especially when considering that specifications are likely written at first in a high-level, incomplete fashion.  This project uses InterpreterLib (\url{http://haskell.org/communities/11-2008/html/report.html#sect5.5.6}) and various Rosetta analysis tools to define a typing analysis that attempts to extract typing information, constraints, and errors to present to the user, in order to guide the specification refinement process.  It is in the early stages of development, but may eventually link up with HaskHOL to discharge some TCC's. For details talk to Mark Snyder.

\item We are developing a library in Haskell for processing Rosetta specifications.
A current focus is the modularity and re-use of distinct processing
elements, such as type-checking, partial evaluation, and reasoning
assistants. Mutually defined elements that are more convenient to
consider as distinct interact via a reactive monadic computation, so
the two elements' code can be managed as separate packages. Also, our
principal specification representation use functors and type-level
fixed points to achieve extensibility and generic programming. The
goal of the library is to provide to a tight and graduated interface
to the basic processing elements, so that the users may incorporate
the most appropriate basic elements when implementing their own, more
domain-specific Rosetta processors. For details talk to Nick Frisby.

\item We are working with other functional programming groups (University of Iowa, St.\ Andrews, Heriot-Watt, Halmstad University, and of course Chalmers) to share our common experiences with using FPGA boards, and generating VHDL.
So far, we have chosen and purchased common Xilinx boards, and have a design for a so called ``$\lambda$-bridge'' between our UNIX invocation infrastructures and our FPGA boards. The idea is we can share experiences, for the sake of 
being able to spend more time working on FP issues, and bringing FP ideas to hardware related problems.
\end{itemize}

\FurtherReading
\begin{compactitem}
\item   The Functional Programming Group:
    \url{http://www.ittc.ku.edu/csdl/fpg}

\item
  CSDL website: \url{https://wiki.ittc.ku.edu/csdl/Main_Page}
\end{compactitem}
\end{hcarentry}
