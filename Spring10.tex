\documentclass{scrreprt}
\usepackage{paralist}
\usepackage{graphicx}
\usepackage{hcar}
\usepackage[all]{xy}    % NOTICE THIS EXTRA PACKAGE


\begin{document}

\begin{hcarentry}{Kansas Lava}
\report{Andy Gill}
\status{Ongoing}
\participants{Andy Gill, Tristan Bull, Andrew Farmer, Garrin Kimmell, Ed Komp, Brett Werling}% optional
\makeheader

Kansas Lava is an attempt to use the Lava design pattern with modern functional programing
technology. In particular we scale up the ideas in Lava to operate
on larger circuits and with large basic components. 
\begin{itemize}
\item Kansas Lava uses a single principal \verb|Signal| type for all types of signals. 
Some versions
of Lava have overloaded signal to interpret signal in either a synthesis or
simulation mode. Our experience is that a single concrete type is easier
to work with in practice, and we have baked the two main interpretations
into our \verb|Signal| type. Ultimately this allows a closer fit between
the specifications of behavior and synthesizable code. 
\item Like other Lava implementations before it, Kansas Lava supports both synthesis and simulation.
The use case would typically be development of a executable model, refinement into
a synthesizable variant, then further refinement for efficiency, and other considerations.
\item Kansas Lava uses a modern FP style of Haskell. We allow \verb|Signal| to
be an applicative functor. Arithmetic is overloaded over Signal,
so we can represent addition using \verb|+|, multiplication using \verb|*|, etc.
Constant literals can also be Signals. 
This leads to a cleaner looking specifications.
\item Kansas Lava has direct support for importing existing
VHDL libraries as new, well typed primitives. This allows Kansas Lava to 
be used as a high-level glue between existing solutions.
\item Kansas Lava includes a simple type checking over binary representations.
Polymorphic specifications inside Lava will be instantiated
to their specific, monomorphically sized implementation in VDHL,
depending on the propagation of actual usage.
\item In Haskell, requiring a 14-bit value is unusual, but in hardware, we often
know and want to enforce a specific width. 
Kansas Lava used an implementation of sized types, built using type functions.
This library, developed specifically for use with Kansas Lava, includes
sized 1 and 2 dimensional matrixes, an sized signed and unsigned bit vectors.
\end{itemize}

We are writing Kansas Lava to address one problem (generate good hardware designs for communication circuits)
and help facilitate a second (explore and understand correctness preserving optimizations of non-trivial
hardware components).
With telemetry circuits, the encodings and decoding mechanisms
are typically express using matrixes operations, so we pay careful
attention to allowing clear encoding of such operations.
In particular, we use type functions to clean up sized types, 
making for a cohesive addition to our Lava.
A release is planned for early November, and will be available on hackage.

\FurtherReading
  \url{http://www.ittc.ku.edu/csdl/fpg/tools/Kansas_Lava}
\end{hcarentry}

\begin{hcarentry}{ChalkBoard}
\report{Andy Gill}
\status{Ongoing}
\participants{Kevin Matlage, Andy Gill}% optional
\makeheader

ChalkBoard is a domain specific language for describing images. 
The language is uncompromisingly functional
and encourages the use of modern functional idioms.
The novel contribution of ChalkBoard is that it uses off-the-shelf
graphics cards to speed up rendering of our functional description.
The intention is that we will use ChalkBoard to animate educational
videos, as well as processing streaming videos.

Here is the basic architecture of ChalkBoard. 
{\small
\[
\xygraph{
!{(-4,0)}{\txt{ChalkBoard\\Image\\Specification}}="CIS",
!{(-4,-1)}*+[F]{\txt{Deep\\DSL}},
!{(0,0)}*+{\txt{ChalkBoard\\IR}}="CBIR",
!{(4,0)}{\txt{OpenGL}}="OpenGL",
!{(4,-1)}*+[F]{\txt{GPU}},
"CIS" :@{->}^{\txt{DSL Capture}}_{\txt{\& Compile}} "CBIR",
"CBIR" :@{->}^{\txt{ChalkBoard}}_{\txt{Back End}} "OpenGL",
!{}
}
\]
}

The image specification language is a deeply embedded Domain Specific Language (DSL).
We capture and compile our DSL, rather than interpret it directly. 
In order to do this, and allow use of a polygon-based back-end,
we have needed to make some interesting compromises,
but the language captured remains pure, has a variant
of functors as a control structure,
and has first-class images.
We compile this language into an imperative intermediate representation
that has first class {\em buffers} -- regular arrays of colors or other entities.
This language is then interpreted by macro-expanding each intermediate representation
command into a set of OpenGL commands.  In this
way, we leverage modern  graphics boards to
do the heavy lifting of the language. 

A release is planned for early November, and will be available on hackage.

\FurtherReading
  \url{http://www.ittc.ku.edu/csdl/fpg/tools/ChalkBoard}
\end{hcarentry}


\end{document}
