\documentclass{scrreprt}
\usepackage{paralist}
\usepackage{graphicx}
\usepackage{hcar}
\usepackage[all]{xy}    % NOTICE THIS EXTRA PACKAGE


\begin{document}

\begin{hcarentry}{Kansas Lava}
\report{Andy Gill}
\status{Ongoing}
\participants{Andy Gill, Tristan Bull, Andrew Farmer, Garrin Kimmell, Ed Komp, Brett Werling}% optional
\makeheader

Kansas Lava is a modern implementation of a hardware description language
that uses functions to express hardware components,
and leverages the abstractions in Haskell to build complex circuits. 
Lava, 
the given name for a family of Haskell based hardware description libraries,
is an idiomatic way of expressing hardware in Haskell which allows for simulation and
synthesis to hardware
Driven by a self-imposed requirement to implementing some specific telemetry circuits in Lava,
we have made a number of recent improvements to both the external API
and the internal representations used. 
We have retained our dual shallow/deep
representation of signals in general, but now have a number of externally visible
abstractions for combinatorial, sequential, and enabled signals.
We also have new abstractions for memory and memory updates.
Internally, we found the need to represent unknown values inside our circuits, 
so we made aggressive use of type functions to lift our values in a principled
and regular way. This design decision unfortunately 
complicates the internals of Kansas Lava, but the external
API remains unaffected.

An overarching design decision is the aggressive use an algebra over {\em commutable functors and observable-functors\/}, the details of which we hope to write up this summer.

We have also been working on a new debugging system, which 

A release is planned for late summer, and will be available on hackage.

\FurtherReading
  \url{http://www.ittc.ku.edu/csdl/fpg/Tools/KansasLava}

Recent presentations about Kansas Lava
\begin{itemize}
\item May 18th, The Internals and Externals of Kansas Lava, Eleventh Symposium on Trends in Functional Programming, Norman, OK.
\item May 11th, Generating Implementations of Error Correcting Codes using Kansas Lava, 10th Annual High Confidence  Software and Systems Conference, Linthicum Heights, MD, Slides, Quicktime Slides.
\item March 11th, Information Assurance, Functional Programming and Kansas Lava, Information Assurance Visit, ITTC.
\item February 26th, Forward Error Correction Codes and Kansas Lava, Brigham Young University, Provo, Utah, jointly with Erik Perrins. 
\end{itemize}

\end{hcarentry}

\begin{hcarentry}{ChalkBoard}
\report{Andy Gill}
\status{Ongoing}
\participants{Kevin Matlage, Andy Gill}% optional
\makeheader

ChalkBoard is a domain specific language for describing images. 
The language is uncompromisingly functional
and encourages the use of modern functional idioms.
The novel contribution of ChalkBoard is that it uses off-the-shelf
graphics cards to speed up rendering of our functional description.
The intention is that we will use ChalkBoard to animate educational
videos, as well as processing streaming videos.

Since the last HCAR report, we've used ChalkBoard for two main projects.
\begin{itemize}
\item Post-processing a ``special feature'' presentation at PEPM, where we turned
a video of KU actors (us) giving a presentation into individual frames,
and processed them using chalkboard to add clearer slides, and some animations.
\item Working on a new animation language, based round a new applicative
functor, \verb|Active|. It has been called Functional Reactive Programming,
without the reactive part!
\end{itemize}

\FurtherReading
  \url{http://www.ittc.ku.edu/csdl/fpg/Tools/ChalkBoard}
\end{hcarentry}

\begin{hcarentry}{Functional Programming at KU}
\report{Andy Gill}
\status{???}
\makeheader

Functional Programming remains active at KU. We continue to 

Put the text here. What's following are suggestions for the content of an entry.

(WHAT IS IT?)

(WHAT IS ITS STATUS? / WHAT HAS HAPPENED SINCE LAST TIME?)

(CAN OTHERS GET IT?)

(WHAT ARE THE IMMEDIATE PLANS?)

\FurtherReading
  \url{(PROJECT URL)}
\end{hcarentry}



\end{document}
