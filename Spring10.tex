%\documentclass{scrreprt}
\documentclass{article}
\usepackage{paralist}
\usepackage{graphicx}
\usepackage{hcar}

\begin{document}

\begin{hcarentry}{Kansas Lava}
\report{Andy Gill}
\status{Ongoing}
\participants{Andy Gill, Tristan Bull, Andrew Farmer, Garrin Kimmell, Ed~Komp}% optional
\makeheader

Kansas Lava is a modern implementation of a hardware description language
that uses functions to express hardware components,
and leverages the abstractions in Haskell to build complex circuits. 
Lava, 
the given name for a family of Haskell based hardware description libraries,
is an idiomatic way of expressing hardware in Haskell which allows for simulation and
synthesis to hardware.

Driven by a self-imposed requirement to implementing some specific telemetry circuits in Lava,
we have made a number of recent improvements to both the external API
and the internal representations used. 
We have retained our dual shallow/deep
representation of signals in general, but now have a number of externally visible
abstractions for combinatorial, sequential, and enabled signals.
We also have new abstractions for memory and memory updates.
Internally, we found the need to represent unknown values inside our circuits, 
so we made aggressive use of type functions to lift our values in a principled
and regular way. This design decision unfortunately 
complicates the internals of Kansas Lava, but the external
API remains unaffected.

An overarching design decision is the aggressive use an algebra over {\em commutable functors and observable functors\/} for circuit refinement, the details of which we hope to write up this summer.

We have also been working on a new debugging system, which combines the deep and 
shallow embedding in a way to allow probes to be inserted onto functions.
The values of the usage of these functions can be observed, as well 
as used to generate test vectors. 

A release is planned for late summer, and will be available on hackage. Recent presentations about Kansas Lava include (slides on website)
\begin{itemize}
\item May 18th, What's the matter with Kansas Lava?, Eleventh Symposium on Trends in Functional Programming, Norman, OK.
\item May 18th, The Internals and Externals of Kansas Lava, Eleventh Symposium on Trends in Functional Programming, Norman, OK.
\item May 11th, Generating Implementations of Error Correcting Codes using Kansas Lava, 10th Annual High Confidence  Software and Systems Conference, Linthicum Heights, MD.
\item February 26th, Forward Error Correction Codes and Kansas Lava, Brigham Young University, Provo, Utah.
\end{itemize}

\FurtherReading
  \url{http://www.ittc.ku.edu/csdl/fpg/Tools/KansasLava}


\end{hcarentry}

\begin{hcarentry}{ChalkBoard}
\report{Andy Gill}
\status{Ongoing}
\participants{Kevin Matlage, Andy Gill}% optional
\makeheader

ChalkBoard is a domain specific language for describing images. 
The language is uncompromisingly functional
and encourages the use of modern functional idioms.
The novel contribution of ChalkBoard is that it uses off-the-shelf
graphics cards to speed up rendering of our functional description.
We always intended to use ChalkBoard to animate educational
videos, as well as processing streaming videos.
Since the last HCAR report, we've used ChalkBoard in two main projects,
covering both these goals.
\begin{itemize}
\item We use ChalkBoard to post-processing a
``special feature'' presentation at PEPM, where we turned
a video of KU actors (err, us) giving a presentation into individual frames,
and processed them using chalkboard to add clearer slides, and some animations.
\item We are working on a new animation language, based round a new applicative
functor, \verb|Active|. It has been called Functional Reactive Programming,
without the reactive part!
\end{itemize}

We talked about a case study of using our \verb|Active| language at TFP in May,
when Kevin gave the talk ``Every Animation Should Have a Beginning, a Middle, and an End''.

\FurtherReading
  \url{http://www.ittc.ku.edu/csdl/fpg/Tools/ChalkBoard}

\end{hcarentry}

\begin{hcarentry}{Functional Programming at KU}
\report{Andy Gill}
\status{Ongoing}
\makeheader

Functional Programming remains active at KU and 
the Computer Systems Design Laboratory in ITTC.
The System Level Design Group (lead by Perry Alexander)
and the Functional Programming Group (lead by Andy Gill)
together form the core functional programming initiative at KU.
Apart from Kansas Lava and ChalkBoard, there are many other
FP and Haskell related things going on.

\begin{itemize}
\item We are developing a Haskell version of HOL.
Traditionally, members of the higher-order logic theorem (HOL) proving family have
been implemented in the Standard ML programming language or one of its derivatives.
HaskHOL aims to break with tradition by implementing a lightweight HOL theorem prover
library as a Haskell hosted domain specific language. Based on the HOL Light logical
system, HaskHOL aims to provide the ability for Haskell users to reason about their
code directly without having to transform it or otherwise export it to an external
tool. For details talk to Evan Austin.

\item We are actively working on enabling {\em Type-Directed Specification Refinement in Rosetta\/}. Rosetta is a specification language that focuses on the interaction between different domains, such as state-based and signal-based domains.  With dependent types, first-class types, and reflection, there are many areas where a traditional all-or-nothing typing analysis would be impractical--especially when considering that specifications are likely written at first in a high-level, incomplete fashion.  This project uses InterpreterLib  and various Rosetta analysis tools to define a typing analysis that attempts to extract typing information, constraints, and errors to present to the user, in order to guide the specification refinement process.  It is in the early stages of development, but may eventually link up with HaskHOL to discharge some TCC's. For details talk to Mark Snyder.

\item We are developing a library in Haskell for processing Rosetta specifications.
A current focus is the modularity and re-use of distinct processing
elements, such as type-checking, partial evaluation, and reasoning
assistants. Mutually defined elements that are more convenient to
consider as distinct interact via a reactive monadic computation, so
the two elements' code can be managed as separate packages. Also, our
principal specification representation use functors and type-level
fixed points to achieve extensibility and generic programming. The
goal of the library is to provide to a tight and graduated interface
to the basic processing elements, so that the users may incorporate
the most appropriate basic elements when implementing their own, more
domain-specific Rosetta processors. For details talk to Nick Frisby.

\item CSDL is developing Oread, a language utilizing monadic concepts capturing message-passing concurrency, to explore the application of functional languages to hardware/software system design and implementation. The Oread toolset, implemented in Haskell,  is used to compile the language to either embedded processor cores or FPGA hardware technology. The compiler is capable of emitting LLVM code, which can then be compiled to the Microblaze soft processor, or VHDL and Verilog, for direct implementation on Xilinx FPGA devices. For details talk to Garrin Kimmell.
\end{itemize}

We also lose Garrin Kimmell in June, when he moves to Iowa.

\FurtherReading
  The Functional Programming Group (with a new website)  
    \url{http://www.ittc.ku.edu/csdl/fpg}.

  CSDL website: \url{https://wiki.ittc.ku.edu/csdl/Main_Page}
\end{hcarentry}


\end{document}
